%
% The first command in your LaTeX source must be the \documentclass command.
\documentclass[sigplan,review]{acmart}

%
% \BibTeX command to typeset BibTeX logo in the docs

% Rights management information. 
% This information is sent to you when you complete the rights form.
% These commands have SAMPLE values in them; it is your responsibility as an author to replace
% the commands and values with those provided to you when you complete the rights form.
%
% These commands are for a PROCEEDINGS abstract or paper.
\copyrightyear{2019}
\acmYear{2019}
\setcopyright{acmlicensed}
\acmConference[Woodstock '18]{Woodstock '18: ACM Symposium on Neural Gaze Detection}{June 03--05, 2018}{Woodstock, NY}
\acmBooktitle{Woodstock '18: ACM Symposium on Neural Gaze Detection, June 03--05, 2018, Woodstock, NY}
\acmPrice{15.00}
\acmDOI{10.1145/1122445.1122456}
\acmISBN{978-1-4503-9999-9/18/06}

%
% These commands are for a JOURNAL article.
%\setcopyright{acmcopyright}
%\acmJournal{TOG}
%\acmYear{2018}\acmVolume{37}\acmNumber{4}\acmArticle{111}\acmMonth{8}
%\acmDOI{10.1145/1122445.1122456}

%
% Submission ID. 
% Use this when submitting an article to a sponsored event. You'll receive a unique submission ID from the organizers
% of the event, and this ID should be used as the parameter to this command.
%\acmSubmissionID{123-A56-BU3}

%
% The majority of ACM publications use numbered citations and references. If you are preparing content for an event
% sponsored by ACM SIGGRAPH, you must use the "author year" style of citations and references. Uncommenting
% the next command will enable that style.
%\citestyle{acmauthoryear}

%
% end of the preamble, start of the body of the document source.
\begin{document}


%
% The "title" command has an optional parameter, allowing the author to define a "short title" to be used in page headers.
\title{How to Assault Your Code with Mad Libs\\for Fun and Profit}

%
% The "author" command and its associated commands are used to define the authors and their affiliations.
% Of note is the shared affiliation of the first two authors, and the "authornote" and "authornotemark" commands
% used to denote shared contribution to the research.
\author{Alex Groce}
\email{agroce@gmail.com}
\affiliation{%
  \institution{School of Informatics, Computing, and Cyber Systems\\Northern Arizona University}
  \streetaddress{1295 S Knoles Dr.}
  \city{Flagstaff}
  \state{Arizona}
  \postcode{86011}
}

\author{David R. MacIver}
\email{david@drmaciver.com}
\affiliation{%
  \institution{Imperial College London}
  \city{London}
  \country{United Kingdom}}

\author{Peter Goodman}
\email{peter@trailofbits.com}
\author{Gustavo Greico}
\email{gustavo.greico@trailofbits.com}
\affiliation{%
  \institution{Trail of Bits}
  \city{New York}
  \state{New York}
}

 

%
% By default, the full list of authors will be used in the page headers. Often, this list is too long, and will overlap
% other information printed in the page headers. This command allows the author to define a more concise list
% of authors' names for this purpose.
%\renewcommand{\shortauthors}{Groce, et al.}

%
% The abstract is a short summary of the work to be presented in the article.
\begin{abstract}
\input{abstract}
\end{abstract}

%
% The code below is generated by the tool at http://dl.acm.org/ccs.cfm.
% Please copy and paste the code instead of the example below.
%
\begin{CCSXML}
<ccs2012>
<concept_id>10011007.10011074.10011099.10011102.10011103</concept_id>
<concept_desc>Software and its engineering~Software testing and debugging</concept_desc>
<concept_significance>500</concept_significance>
</concept>
</ccs2012>
\end{CCSXML}

\ccsdesc[500]{Software and its engineering~Software testing and debugging}
%
% Keywords. The author(s) should pick words that accurately describe the work being
% presented. Separate the keywords with commas.
\keywords{automated test generation, parsing, parameterized unit testing}

%
% A "teaser" image appears between the author and affiliation information and the body 
% of the document, and typically spans the page. 


%
% This command processes the author and affiliation and title information and builds
% the first part of the formatted document.
\maketitle

\section{Introduction: What is a Mad Lib?}

\section{The Simplest Random Tester}

\section{Everything is Parsing}

\subsection{A Test Generator as an Interpreter}

\section{Tools that Work This Way}

One such tool is Hypothesis \cite{hypothesis}.  Another is DeepState \cite{DeepState}.

\section{Practical Implications:  Implementing Ranges in DeepState}



% The next two lines define the bibliography style to be used, and the bibliography file.
\bibliographystyle{ACM-Reference-Format}
\bibliography{bibliography}

% 

\end{document}
